\startchapter[title={A Calligraphy Grid in ConTeXt/LMTX},reference={a-calligraphy-grid-in-contextlmtx}]

This project is a generator of grids for medieval calligraphy in \goto{\ConTeXt/LMTX}[url(https://wiki.contextgarden.net/)] and \goto{\MetaPost}[url(https://wiki.contextgarden.net/MetaPost)].

\startsection[title={Files included},reference={files-included}]

\startitemize[packed]
\item
  Main module:\crlf
  \type{tex/context/third/pauta/t-pauta.mkxl}
\item
  Example file:\crlf
  \type{doc/context/third/pauta/pauta-example.tex} (also in pdf)
\item
  A copy of this file in \ConTeXt~format:\crlf
  \type{doc/context/third/pauta/pauta-example.tex} (also in pdf)
\item
  Environment file for the examples (not needed for Pauta to work):\crlf
  \type{doc/context/third/pauta/env-pauta.tex}
\item
  This \type{README.md} file, and a pandoc generated ConTeXt version at \type{doc/context/third/pauta/README.tex}. It gets regenerated by the \type{build.lua} script on each run using pandoc.
\item
  The build script \type{build.lua}, that takes care of generating the tex version of the README file and all the pdfs.
\stopitemize

\stopsection

\startsection[title={Use},reference={use}]

\startenumerate[n,packed][stopper=.]
\item
  Clone this repository: \type{git clone https://github.com/conradolandia/pauta.git}, and \type{cd} inside it.
\item
  Copy the \type{doc} and \type{tex} folders to your \ConTeXt~tree and rebuild your database with \type{context --generate}. You can find more details about the process \goto{on the \ConTeXt~wiki}[url(https://wiki.contextgarden.net/Modules\#Installation)]. Alternatively, call context with the \type{--path} flag, and provide it with the path of this folder, i.e: \type{context --path=/home/user/pauta}. Alternatively still, simply place \type{t-pauta.mkxl} on the same directory as the file importing it.
\item
  Invoke the \type{\Pauta} macro as many times as you want pages. Each invocation can have a different configuration. Each invocation will create one single page.
\item
  The data about the hand\footnote{This a way of saying the {\em font} in fancy calligraphic terms.} is autogenerated by the module and set into the header or the footer, following the user configuration.
\item
  {\bf Warning:} This module takes over the top and bottom typesetting areas\footnote{Check out the \goto{layout}[url(https://wiki.contextgarden.net/Page_Layout)] article in the wiki for more information about \ConTeXt~typesetting areas.}, and does not reset them properly yet. So if your document includes other content on those areas you will need to reset again to your liking by manually invoking \type{\setuptoptexts}/\type{\setupbottomtexts}. This will be hopefully improved in the future.
\stopenumerate

\stopsection

\startsection[title={Generating the example file},reference={generating-the-example-file}]

Review and run the lua script \type{build.lua}. If you don't have a standalone lua interpreter, you can run it with luametatex like so:

\starthighlighting
/BTEX\ExtensionTok{luametatex} \AttributeTok{--luaonly}\NormalTok{ build.lua}/ETEX
\stophighlighting

You can also adapt it and use to build your own projects, \type{build.lua} lives in its own \goto{repository}[url(https://github.com/conradolandia/build.lua/)].

\stopsection

\startsection[title={Configuration Parameters},reference={configuration-parameters}]

All parameters are optional. Defaults are as follows:

\starthighlighting
/BTEX\FunctionTok{\letterbackslash{}Pauta}\NormalTok{[}/ETEX
/BTEX\NormalTok{ { }hand=, }\CommentTok{\letterpercent{} Hand name. If not defined, will not show info on the right of the header \letterslash{} footer}/ETEX
/BTEX\NormalTok{ { }handInfo=, }\CommentTok{\letterpercent{} Some extra info for the hand. If not defined, will not show info on the right of the header \letterslash{} footer}/ETEX
/BTEX\NormalTok{ { }infoPosition=header, }\CommentTok{\letterpercent{} Where to show the extra info (header \letterbar{} footer)}/ETEX
/BTEX\NormalTok{ { }infoLeft=\letteropenbrace{}}\FunctionTok{\letterbackslash{}setup}\NormalTok{\letteropenbrace{}pauta:content:leftmark\letterclosebrace{}\letterclosebrace{}, }\CommentTok{\letterpercent{} If defined, will override autogenerated hand info on the left of the footer \letterslash{} header}/ETEX
/BTEX\NormalTok{ { }infoRight=\letteropenbrace{}}\FunctionTok{\letterbackslash{}setup}\NormalTok{\letteropenbrace{}pauta:content:rightmark\letterclosebrace{}\letterclosebrace{}, }\CommentTok{\letterpercent{} If defined, will override autogenerated hand info on the right of the footer \letterslash{} header}/ETEX
/BTEX\NormalTok{ { }displayNibs=false, }\CommentTok{\letterpercent{} Show nib-width marks (true \letterbar{} false)}/ETEX
/BTEX\NormalTok{ { }displayAngleMarks=false, }\CommentTok{\letterpercent{} Display dotted guides for the nib angle (true \letterbar{} false)}/ETEX
/BTEX\NormalTok{ { }nibWidth=3mm, }\CommentTok{\letterpercent{} Pen nib width (must include units, or it will default to big points)}/ETEX
/BTEX\NormalTok{ { }nibAngle=35, }\CommentTok{\letterpercent{} Nib working angle in degrees}/ETEX
/BTEX\NormalTok{ { }ascenders=3, }\CommentTok{\letterpercent{} Number of ascender lines (in nib widths)}/ETEX
/BTEX\NormalTok{ { }xHeight=4, }\CommentTok{\letterpercent{} Number of x-height lines (in nib widths)}/ETEX
/BTEX\NormalTok{ { }descenders=3, }\CommentTok{\letterpercent{} Number of descending lines (in nib widths)}/ETEX
/BTEX\NormalTok{ { }adjustment=0, }\CommentTok{\letterpercent{} Sometimes it's necessary to adjust the line height if is longer than TextHeight, still not sure why it happens but it happpens... a value of 1 or 2 should solve it.}/ETEX
/BTEX\NormalTok{ { }mainColor=\letteropenbrace{}s=.4\letterclosebrace{}, }\CommentTok{\letterpercent{} Main color (lines that separate sections)}/ETEX
/BTEX\NormalTok{ { }secondaryColor=\letteropenbrace{}s=.6\letterclosebrace{}, }\CommentTok{\letterpercent{} Secondary color (lines separated by a nib width and dotted angle lines)}/ETEX
/BTEX\NormalTok{ { }tertiaryColor=\letteropenbrace{}s=.8\letterclosebrace{}, }\CommentTok{\letterpercent{} Tertiary color (nib width marks on the left margin)}/ETEX
/BTEX\NormalTok{]}/ETEX
\stophighlighting

\stopsection

\startsection[title={Code Examples},reference={code-examples}]

\startsubsection[title={Example 1: Basic Usage},reference={example-1-basic-usage}]

\starthighlighting
/BTEX\FunctionTok{\letterbackslash{}usemodule}\NormalTok{[pauta]}/ETEX

/BTEX\FunctionTok{\letterbackslash{}startdocument}/ETEX
/BTEX\FunctionTok{\letterbackslash{}Pauta}\NormalTok{[}/ETEX
/BTEX\NormalTok{ { }hand=\letteropenbrace{}Carolingian\letterclosebrace{},}/ETEX
/BTEX\NormalTok{ { }handInfo=\letteropenbrace{}Tours school, VIII}\FunctionTok{\letterbackslash{}high}\NormalTok{\letteropenbrace{}th\letterclosebrace{} century\letterclosebrace{},}/ETEX
/BTEX\NormalTok{ { }infoPosition=header,}/ETEX
/BTEX\NormalTok{ { }displayNibs=true,}/ETEX
/BTEX\NormalTok{ { }displayAngleMarks=true,}/ETEX
/BTEX\NormalTok{ { }nibWidth=3mm,}/ETEX
/BTEX\NormalTok{ { }ascenders=2,}/ETEX
/BTEX\NormalTok{ { }xHeight=3,}/ETEX
/BTEX\NormalTok{ { }descenders=2,}/ETEX
/BTEX\NormalTok{ { }adjustment=0,}/ETEX
/BTEX\NormalTok{ { }mainColor=\letteropenbrace{}s=.6\letterclosebrace{},}/ETEX
/BTEX\NormalTok{ { }secondaryColor=\letteropenbrace{}s=.8\letterclosebrace{},}/ETEX
/BTEX\NormalTok{ { }tertiaryColor=\letteropenbrace{}s=.8\letterclosebrace{},}/ETEX
/BTEX\NormalTok{]}/ETEX
/BTEX\FunctionTok{\letterbackslash{}stopdocument}/ETEX
\stophighlighting

\stopsubsection

\startsubsection[title={Example 2: Multiple Pauta Instances},reference={example-2-multiple-pauta-instances}]

\starthighlighting
/BTEX\FunctionTok{\letterbackslash{}usemodule}\NormalTok{[pauta]}/ETEX

/BTEX\FunctionTok{\letterbackslash{}startdocument}/ETEX
/BTEX\FunctionTok{\letterbackslash{}Pauta}\NormalTok{[}/ETEX
/BTEX\NormalTok{ hand=\letteropenbrace{}Carolingian\letterclosebrace{},}/ETEX
/BTEX\NormalTok{ handInfo=\letteropenbrace{}Tours school, VIII}\FunctionTok{\letterbackslash{}high}\NormalTok{\letteropenbrace{}th\letterclosebrace{} century\letterclosebrace{},}/ETEX
/BTEX\NormalTok{ infoPosition=header,}/ETEX
/BTEX\NormalTok{ displayNibs=true,}/ETEX
/BTEX\NormalTok{ displayAngleMarks=true,}/ETEX
/BTEX\NormalTok{ nibWidth=3mm,}/ETEX
/BTEX\NormalTok{ ascenders=2,}/ETEX
/BTEX\NormalTok{ xHeight=3,}/ETEX
/BTEX\NormalTok{ descenders=2,}/ETEX
/BTEX\NormalTok{ adjustment=0,}/ETEX
/BTEX\NormalTok{ mainColor=\letteropenbrace{}s=.5\letterclosebrace{},}/ETEX
/BTEX\NormalTok{ secondaryColor=\letteropenbrace{}s=.6\letterclosebrace{},}/ETEX
/BTEX\NormalTok{ tertiaryColor=\letteropenbrace{}s=.7\letterclosebrace{},}/ETEX
/BTEX\NormalTok{]}/ETEX

/BTEX\CommentTok{\letterpercent{} Overriding the hader \letterslash{} footer info:}/ETEX

/BTEX\FunctionTok{\letterbackslash{}Pauta}\NormalTok{[}/ETEX
/BTEX\NormalTok{ infoLeft=\letteropenbrace{}An excercise in Visigothic script\letterclosebrace{},}/ETEX
/BTEX\NormalTok{ infoRight=\letteropenbrace{}from an Spanish manuscript, VII}\FunctionTok{\letterbackslash{}high}\NormalTok{\letteropenbrace{}th\letterclosebrace{} century\letterclosebrace{},}/ETEX
/BTEX\NormalTok{ infoPosition=footer,}/ETEX
/BTEX\NormalTok{ displayNibs=true,}/ETEX
/BTEX\NormalTok{ displayAngleMarks=false,}/ETEX
/BTEX\NormalTok{ nibWidth=2mm,}/ETEX
/BTEX\NormalTok{ ascenders=4,}/ETEX
/BTEX\NormalTok{ xHeight=3,}/ETEX
/BTEX\NormalTok{ descenders=4,}/ETEX
/BTEX\NormalTok{ adjustment=1,}/ETEX
/BTEX\NormalTok{ mainColor=\letteropenbrace{}s=.3\letterclosebrace{},}/ETEX
/BTEX\NormalTok{ secondaryColor=\letteropenbrace{}s=.4\letterclosebrace{},}/ETEX
/BTEX\NormalTok{ tertiaryColor=\letteropenbrace{}s=.5\letterclosebrace{},}/ETEX
/BTEX\NormalTok{]}/ETEX
/BTEX\FunctionTok{\letterbackslash{}stopdocument}/ETEX
\stophighlighting

\stopsubsection

\stopsection

\stopchapter
